\chapter{Conclusions}
\label{chap:conclusions}
\minitoc \mtcskip \noindent

\section{Overview}

All the work and contributions of this dissertation aim towards proving whether the following statement is valid or not: \textbf{\textit{"Blockchain a good architectural design for the Supply Chain Management domain."}}

 For this reason, 3 questions were formulated in the problem statement:
\begin{enumerate}
    \item \textbf{What supply chain issues, improvements and requirements do the experts really find the most important?}
    \item \textbf{What is the Blockchain tool or framework most adequate to the development of an architecture that can support these requirements?}
    \item \textbf{Is it possible to build a feasible architectural design, by using such a tool, to implement all these requirements?}
\end{enumerate}
    
Chapter \ref{chap:survey} provided the answer to the first question. Chapter \ref{chap:prototype} used this answer to provide the answer to the second and third questions, which can now lead to a final conclusion towards the initial statement.


\subsection*{Thesis Statement}
\par We have almost all of the pieces to reach a final conclusion towards the main thesis statement. However, there is still the need for the expression "a good architectural design for the SCM domain" to be clarified. \textbf{What exactly does it mean for an architectural design to be good for some domain?} 

If we were to compare a new architectural design against other existing designs for that same domain, then a new design could be considered good if it could cover all of the functionalities of the other designs and have some additional ones or be more efficient. But it was already determined that \textbf{the focus of this thesis statement was not to compare this architecture to others, but to evaluate it objectively and not relatively}. Therefore, a good design could be one that objectively covers all of the most important elicited requirements.

\par The chosen framework had the possibility of satisfying the most requirements, out of all the frameworks. Even so, by answering the last question, it was concluded that the proposed design, even while choosing the most convenient tool possible, could only fill the requirements partially, with some important requirements not being fully achievable at the present moment.


\emph{Therefore, if a good design is one that must be able to satisfy all of the requirements, then Blockchain is not a good architecture for the supply chain management domain.}


However, this does not mean Blockchain is not useful, since it can still partially fill some important requirements. \textbf{The fact that by itself, it might not good be enough to satisfy all the requirements, does not mean it cannot be used together with the current architectures} to fill in some gaps. Blockchain can still be applied as an aid and enhancement to some information system functionalities. For instance, financial transactions was a requirement that was fulfilled and listed very high on the requirements from the survey. No other present architecture uses this functionality, but through Blockchain, it can be implemented.

\section{Contributions}
It is important that a conclusion was reached, but it is also important to look at what work was done to reach that conclusion. The whole process of originating the answers to the questions that support the conclusion also generated useful contributions. 

These contributions might be useful for future research:

\begin{itemize}
	\item Survey Results and Analysis of
	\begin{itemize}
		\item List of the most important supply chain issues, supply chain points of improvement, information system features that a supply chain needs and the most wanted \textbf{blockchain use cases for supply chain management}.
		\item List of requirements for a blockchain-based supply chain management software, grouped by area.
	\end{itemize}
	\item List of Framework requirements and framework comparison.
	\item Architectural design
	\item Prototype
\end{itemize}

\todo{fcorreia: I'd expect to see here a list of contributions of your thesis. You do mention them, but a) I think they would be clearer to read as a list, b) you may be currently missing a few. I see at least 3 or 4, for example: a study of important design considerations for SCMSs, a study of important design considerations for blockchains, a prototype, an analysis of the benefits and liabilities of supporting a SCMS with a blockchain}


\section{Difficulties}
%IMPORTANT
%Maybe move to the approach
Mention the initial thing we wanted to do: to use some metrics to compare our implementation with others from the state of the art.
Mention why we could not do it: NO ONE ANSWERED WITH THE METRICS :( Contacts did not go through as expected!

% Nao ha grande baseline para comparar; há que restringir o ambito e o estudo foi feito para ser baseado nestas consideraçoes; 

%These conclusions are for PDIS; so they will close this delivery and may anticipate the expected results of the dissertation. Include the tasks and a Gantt diagram with the plan.
The dissertation focuses on finding out how applicable Blockchain is to the supply chain industry and to SCM, and there are many variables to consider. Many companies have already started working on projects similar to what is trying to be accomplished here, though with a higher focus on financial benefits than on the research itself (many of the examples given were of companies using Ethereum and its tokens). These solutions have not yet been proved and are also taking their initial steps. The main focus of this dissertation is to develop a proof-of-concept that indeed Blockchain can improve SCM in some form, and also give an insight on how this might be achieved. 

\section{Future Work}

Future work: The work here done was much more introductory in nature, and explored the requirements and their satisfaction as a whole. Additionaly, the maturity of the tools used was questionable. 
Future work might include:
\begin{itemize}
	\item Taking specific requirements from the most important requirements on the list and research on using blockchain to successfully apply them as enhancements:
	\begin{itemize}
		\item Financial applications, payments and contractual agreements;
		\item "Applying Blockchain as a financial alternative in the Supply Chain" could even be a new thesis derived out of this one;
	\end{itemize}
	\item Research and attempt to use different frameworks to fulfill the same requirements that were elicited in this dissertation;
    \item Research on different architectures
\end{itemize} 

%Nao usar situaçoes hipoteticas
%USAR OUTRAS FERRAMENTAS
%COMPARAR COM CLOUD E OUTRAS ARQUITETURAS
%VALIDAR PROTOTIPO COM PESSOAS REAIS

%The research for this dissertation is divided into six steps, described in Table \ref{table:gantt_chart} and illustrated by Figure \ref{fig:gantt_chart}. The research started with the literature review, which presents some important insights, frameworks as well as some important design aspects and decisions to take while developing a model for a Blockchain-driven supply chain project.
%
%These aspects will be taken into account for the next steps. First, adapting and creating a Blockchain integration model, by making some decisions on the design and frameworks to be used. Then, a small integration project will be built based on this model.

%In the final part of the project, its applicability and performance will be evaluated. The developed system itself is a proof-of-concept, which, already proves that such a concept might work, but it is important to measure how it fares against the traditional systems used in supply chain. This evaluation is done in two different parts: the first consists on evaluating functionality, and checking which functionalities are added or subtracted by using blockchain; the second part consists in using performance metrics to evaluate this dissertation's solution against the baseline of a traditional solution, like a centralized system. The metrics used to evaluate this include, but are not limited to:
%\begin{itemize}
%\item Throughput - processed transactions per second
%\item Latency - average time to process a single transaction
%\item Latency volatility - measure of the variety of latency
%\item Security - qualitative evaluation that includes items such as immutability, denial of service resilience, trust and fraud protection, confidentiality and access control
%\item Hardware requirements - how much hardware is needed, and how powerful
%\item Scalability - number of nodes, transactions, users and how much the system can stretch these numbers
%\end{itemize}

%\todo{fcorreia: do you have already any additional ideas about the baseline that you will be comparing to? it'd be great if you could provide a bit more detail about this; will it be a non-blockchain-based distributed system? (e.g., a microservice-based system, or a system based on a replicated datastore) Will it be simply a "normal" centralized system? Depending on the baseline that you choose the conclusions that you will be able to reach will be quite different}

%Finally, the model might have to be recalibrated, taking into account the results gathered from these metrics. The values of the model might have to be changed iteratively, until a satisfying solution is achieved, if possible.


%\section{Expected Results}
%From this dissertation, it is expected that a proof-of-concept system with certain characteristics will emerge. The results will be analyzed to check if the system follows the parameters we are looking for.

%For instance, it should be able to integrate information from various sources into a single place. This information should be available at any time, anywhere. It should be cryptographically secure, immutable, and easy and quick to share.

%In the end, we need to analyze the results with the given metrics and ascertain whether there is a significant increase in performance and functionality that would justify using blockchain in a real supply chain.
