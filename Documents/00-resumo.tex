\chapter{Resumo}
% Introduction
Empresas de qualquer indústria trocam informação e recursos entre elas, criando um fluxo comummente chamado de cadeia de logística. A direção deste fluxo vai do fornecedor até ao consumidor, com trocas e transformações complexas a acontecer desde a origem até à entrega final do produto. Assim, a gestão da cadeia de logística é essential para coordenar esta cadeia.

%Problem
A cadeia de logística enfrenta muitas dificuldades: rastreabilidade de produtos, gestão de inventário, controlo de qualidade e de prazos são apenas algumas. Atrasos são comuns e podem afetar as finanças, crescimentoe reputação de uma empresa. A adicionar a estas dificuldades, muitas vezes a informação necessária a certos processos não está disponível ou não é exata, consequencia, em parte, dos processos manuais usados para inserir informação nos sistemas, o que é lento e pode levar a erros. Isto poderá ser causado pela não existencia de tecnologias de confiança que possam integrar toda a informação de forma segura e rápida.

Uma forma de abordar estes problemas da cadeia de logística é priorizá-los e atualizar as tecnologias de suporte da cadeia de logística. Uma tecnologia que parece adequada é a arquitetura distribuída de blockchain. Esta é uma tecnologia distribuida que permite o armazenamento seguro e imutável de dados, levando a que a informação esteja acessível em qualquer lugar, a qualquer hora. Assim, as blockchains podem ser o meio mais adequado de atingir a rastreabilidadede uma cadeia de logística, possivelmente levando a que esta se torne completamente digital.

Esta dissertação foca-se em pesquisar os problemas da gestão de cadeias de logística e a que nível poderão ser aplicadas blockchains de forma benéfica para resolver esses problemas. A hipótese é que as blockchain possam ser uma arquitetura adequada para sistemas de gestão das cadeias de logística.

Para validar estes problemas e levantar requisitos para um sistema de software de logística, foi realizado um inquérito. Usando estes requisitos, foi possível propor e iterativamente construir uma arquitetura de blockchain, para testar se os requisitos eram possíveis de ser implementados.

No final, o sistema proposto foi analisado e comparado com os requisitos, para verificar se a implementação foi possível. Daqui foi possível tirar algumas conclusões, nomeadamente que esta arquitetura provou conseguir cumprir a maior parte dos requisitos, mas a implementação de outros não foi conseguida e permanece por provar que consiga ser feita. Assim, espera-se que as contribuições do trabalho aqui realizado possam ser uma base para trabalhos futuros de investigação em integrar esta arquitetura com sistemas reais de gestão de cadeias de logística, nomeadamente a partir dos requisitos implementados.

%\chapter{Resumo}
%Tradução PT do abstract




